Chlorophyll a fluorescence images were captured once per hour during the daylight period in a DEPI chamber [CITE].  A set of 5 images (for averaging) were captured using a Hitachi KP-F145GV CCD (Hitachi Kokusai Electric America Inc., Woodbury, NY)  camera outfitted with an infrared long pass filter (Schott Glass RG-9, Thorlabs, Newton, NJ), during a short period ($<400 msec$) of intense light saturating to photosynthesis ($>10,000 \mu mol~photons~m^{-2} s^{-1}$) provided by an array of white Cree LEDs (XMLAWT, 5700K color temperature, Digi-Key, Thief River Falls, MN) collimated using a 20mm Carclo Lens (10003, LED Supply, Lakewood, CO).
%
Chlorophyll a fluorescence was excited using monochromatic red LEDs (Everlight 625nm, ELSH-F51R1-0LPNM-AR5R6, Digi-Key), collimated using a Ledil reflector optic ($C11347\_REGINA$, Mouser Electronics, Mansfield, TX) and pulsed for $50 \mu s$ during a brief window when the white LEDs were electronically shuttered.  A series of 5 images were also collected in the absence of excitation light for artifact subtraction.

Infrared images were collected once per hour with the same camera and filter used for chlorophyll fluorescence.  Pulses of $940 nm$ light were provided by an array of OSRAM LEDs (SFH 4239, Digi-Key), collimated using a Polymer Optics lens (Part no. 170, Polymer Optics Ltd., Berkshire, England).  Since $940 nm$ light does not influence plant development or drive photosynthesis, images were also collected during the night period.  Sets of 15 images were collected for averaging, in the absence of saturating illumination.   As with chlorophyll a fluorescence, images were captured in the absence of $940 nm$ light for artifact subtraction.

[cite]Cruz, J.A. et al. Dynamic Environmental Photosynthetic Imaging (DEPI) Reveals Emergent Phenotypes Related to the Environmental Repsonses of Photosynthesis. Nature Biotechnology Submitted (2015).
