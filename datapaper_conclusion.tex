\section{Conclusion and Discussion}

This paper presents a newly collected multi-modality plant imagery database, ``MSU-PID''.
Compared to existing databases in the field, MSU-PID not only has multiple calibrated modalities, but also enables a wide variety of plant image analysis applications.
Therefore, we believe this new database will be beneficial to the research community in terms of algorithm development, performance evaluation, and identifying new research problems in plant image analysis.
Furthermore, we are also open to suggestions and comments from the users of this database to further enhance our imaging setup and capturing protocol, so that we can develop new databases in the future.

It might be noticed that all plants (respectively for Arabidopsis and bean) in MSU-PID belong to the same genotype and no treatments applied. %This would entail that the proposed dataset cannot be used to investigate computer vision algorithms or imaging modalities in relation to group differences. Could the authors comment on this? This is true. We are not comparing group differences, based on morphology (for example).
This is because we think a fundamental issue with visual phenotyping, i.e. accurate and automated identification and tracking of individual leaves over developmental time scales (weeks), which importance has being highlighted in Introduction, needs to be solved before expending our goal to define group differences. The inherent challenge in this issue is that as leaves emerge and grow, they change in size, position and shape and they may overlap or be overlapped by other leaves.  %I think that new text we have added more clearly emphasizes leaf identification and tracking over time rather than developing methods/algorithms to define group differences.


