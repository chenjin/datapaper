\section{Prior Work} 
\label{sec:prior}

Databases drive computer vision research.
Hence, it is always important to develop and promopt properly captured databases in the vision community.
While there is a clear desire to apply computer vision to plant imagery, the lack of databases is an obstacle for the further study and development.


We summarize all existing publicly available databases that are related to plant imagery in Table~\ref{tab:database}.
In terms of potential applications of the databases, they can be categorized into two types.
The first type is for the general purpose of recognizing a particular species of tree or plant.
The Swedish leaf database~\cite{soderkvist2001computer} is probably the first leaf database even though the images are captured by scanners. 
The Flavia database~\cite{wu2007leaf} is consideraly larger and a neural network is used to train a leaf classifier.
The most recent leafsnap project is an impressive effort that includes a very large dataset of $184$ tree types~\cite{kumar2012leafsnap}.
A mobilephone app is also developed to make the leaf classification system portable.
Finally, the crop/weed image database~\cite{haug2014crop} is captured by a robot in the real field, and used for classifying crop vs.~weed.
Note that in this type of databases, normally only a single leaf is imaged and as a result, the challenging problem of leaf segmentation has been bypassed.


The second type of databases is for plant phenotyping where it is important to keep the plant live while imaged.
Thus, no destructive action is taken and an entire plant is imaged.
The LSC database~\cite{scharr2014annotated} is the most similar one to our database.
It captures a large set of RGB images for the Arabidopsis and Tobacco plants.
The provided manual labels allow the evaluation of leaf segmentation and leaf counting.
In comparison, our MSU-PID database utilizes four sensing modalities in the data capturing, each providing different aspects of plant appearance.
Our manual label also enable us to develop algorithm on additional applications such as leaf tracking and alignment. 

\begin{table}[t!]
	\centering
	\caption{Plant image databases, where the abbreviation in ``Applications'' column is defined as Leaf Classification (LC), Leaf Segmentation (LS), Leaf Counting (LO), Leaf Alignment (LA), and Leaf Tracking (LT).}
	\resizebox{12cm}{!} {
	%\begin{tabular}{|l@{}|@{}c@{}|@{}c@{}|@{}c@{}|@{}c@{}|@{}c@{}|}
	\begin{tabular}{l|c|c|c|c|c|c}
		\hline
		\multirow{2}{*}{Database}&  \multirow{2}{*}{Modality}  & Appli-& Plant & Subject/ &Total  &  Labeled   \\ 						
			&         & cations &            Type         &       Classe \#         &  Image \#    & Image \# \\ \hline
		
Swedish leaf &  Scaned leaf & LC& Swedish trees & $15$ & $1,125$ & $1,125$ \\ \hline
Flavia& RGB  & LC& Leaves & $32$  & $2,120$ & $2,120$ \\ \hline
Leafsnap  & RGB & LC& USA trees & $184$ & $29,107$  &  $29,107$ \\ \hline
Crop/weed &  RGB &Weed det. & Crop/Weed & $2$  & $60$ & $60$ \\ \hline
\multirow{2}{*}{LSC} & \multirow{2}{*}{RGB}  & \multirow{2}{*}{LS, LO} & Arabidopsis &  $43$ & $6287$ & $201$ \\ \cline{4-7}
							    &  & & Tobacco & $80$ & $165,120$ & $83$ \\ \hline
\multirow{2}{*}{MSU-PID}  & Fluorescence,  & LS, LO, & Arabidopsis &  $20$ & $XXX$ & $XXX$ \\ \cline{4-7}
							    & IR, RGB, Depth & LA, LT & Bean & $5$ & $XXX$ & $XXX$ \\ \hline
	       \hline
	\end{tabular}
	}
	\label{tab:database}
\end{table}