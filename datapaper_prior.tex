\section{Prior Work}
\label{sec:prior}

Databases drive computer vision research. 
Hence, it is always important to develop and promote properly captured databases in the vision community. 
While there is a clear desire to apply computer vision to plant image analysis, the lack of publicly available plant image databases has been an obstacle for the further study and development.

We summarize all existing publicly available databases that are related to plant imagery in Table~\ref{tab:database}. In terms of potential applications of these databases, they can be categorized into two types. 
The first type is for the general purpose of recognizing a particular species of tree or plant. 
The Swedish leaf database~\cite{soderkvist2001computer} is probably the first leaf database even though the images are captured by scanners. 
The Flavia database~\cite{wu2007leaf} is considerably larger and a neural network is utilized to train a leaf classifier. 
The most recent leafsnap project is an impressive effort that includes a very large dataset of leaves for $184$ tree types~\cite{kumar2012leafsnap}. 
A mobile phone application is also developed to make the leaf classification system portable. Finally, the crop/weed image database~\cite{haug2014crop} is captured by a robot in the real field, and used for the classification of crop vs.~weed. 
Note that in this type of databases, normally only a {\it single} leaf is imaged in a relatively constrained imaging environment and as a result, the challenging problem of leaf segmentation has been bypassed.

The second type of databases is for plant phenotyping, where it is important to capture plant images without interfering the growth of plants. 
Thus, non-destructive imaging approaches are taken and the entire plant is imaged. The LSC database~\cite{scharr2014annotated} is the most relevant one to our database. It captures a large set of RGB images for the Arabidopsis and Tobacco plants. The provided manual labels allow the evaluation of leaf segmentation and leaf counting. In comparison, our MSU-PID database utilizes four sensing modalities in the data capturing, each providing different aspects of plant visual appearance.
Our diverse manual labels also enable us to develop algorithms for additional applications such as leaf tracking and leaf alignment.
