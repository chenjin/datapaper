\section{Baseline Method and Performance}
\label{sec:baseline}

To facilitate future research on this database, we apply our automatic multi-leaf segmentation, alignment, and tracking framework~\cite{yin2014a,yin2014b} to the testing set of Arabidopsis imagery to provide a baseline.
Specifically, we apply it on fluorescence and RGB modalities. 
Note that~\cite{yin2014a,yin2014b} is designed for rosette plants like Arabidopsis.
Therefore, it will not be applied to bean plant as it does not belong to rosette plants.
%In addition, we also apply it on the RGB modality and use the propagated labels for evaluation. 

\subsection{Multi-leaf Segmentation and Tracking Framework}
As shown in Figure~\ref{fig:methodOverview}, the input of this framework is a plant video and a set of predefined templates with various shapes, scales, and orientations.
We treat all images in $9$ days as a video from first image on the first day to the last image on the last day.
To generate the template set, we first select $12$ templates with different aspect ratios from the labeled images in the training set together with the corresponding tip locations.
For each template, we scale it to $10$ different sizes in order to cover the entire range of leaf sizes in the database.
The scales for fluorescence images and RGB images are different due to the different image resolutions. 
For each scale template, we rotate every $15^{\circ}$ to generate $24$ templates at different orientations.
Tip locations will be scaled and rotated accordingly.
Finally, we generate $2,880$ leaf templates.

Our work is motivated by Chamfer Matching (CM) technique~\cite{barrow1977parametric}, which is used to align two edge maps.
We extend it to simultaneously align multiple overlapping objects.
For plant segmentation, we use simple thresholding process and edge detection to generate an image mask and edge map.
The best threshold is learnt from the training set, which is done by tuning the threshold in a certain range and find the best one by evaluating the overlap of the segmented masks with the groundtruth label masks.
The edge map and mask are used in the alignment and tracking optimization.

First, we find the best location of each template in the edge map that has the minimal CM distance, which will result in an over-completed set of leaf candidates.
Second, we apply multi-leaf alignment~\cite{yin2014a} approach to find an optimal set of leaf candidates on the last frame of the video, which will provide the information of the number of leaves, tip locations and boundaries of each leaf.
Third, we apply multi-leaf tracking~\cite{yin2014b} approach, which is based on leaf template transformation, to track leaves between continuous two frames.

In the tracking process, we delete a leaf when it becomes too small.
We develop a procedure to generate new leaves when there is a relatively large portion of the mask has not been covered by current leaf candidates.
For each frame of the video, we can generate a label image with each leaf being labeled with one color and the tip locations for each estimated leaf.
The labeled color for each leaf in the video remains the same during the tracking process.


\begin{figure*}
\begin{centering}
\begin{tabular}{c c c c c c}
%\begin{tabular}{lllllllll}
\includegraphics[width=.11\textwidth]{Figures/AlignPerformance/9_1}&
\includegraphics[width=.11\textwidth]{Figures/AlignPerformance/10_1}&
\includegraphics[width=.11\textwidth]{Figures/AlignPerformance/11_1}&
\includegraphics[width=.11\textwidth]{Figures/AlignPerformance/12_1}&
\includegraphics[width=.11\textwidth]{Figures/AlignPerformance/14_1}&
\includegraphics[width=.11\textwidth]{Figures/AlignPerformance/15_1}\\

\includegraphics[trim= 90 60 100 5, clip, width=.11\textwidth]{Figures/AlignPerformance/9_2}&
\includegraphics[trim= 90 60 100 5, clip, width=.11\textwidth]{Figures/AlignPerformance/10_2}&
\includegraphics[trim= 90 60 100 5, clip, width=.11\textwidth]{Figures/AlignPerformance/11_2}&
\includegraphics[trim= 90 60 100 5, clip, width=.11\textwidth]{Figures/AlignPerformance/12_2}&
\includegraphics[trim= 90 60 100 5, clip, width=.11\textwidth]{Figures/AlignPerformance/14_2}&
\includegraphics[trim= 90 60 100 5, clip, width=.11\textwidth]{Figures/AlignPerformance/15_2}\\
(a) & (b) & (c) & (d) & (e) & (f) \\
\end{tabular}
\caption{Leaf alignment results on the last frame of $6$ fluorescence Arabidopsis videos. First row shows the original images. Second row shows the alignment results with red points denoting the boundary of the leaf templates in the blue bounding boxes. The numbers on the leaves are the leaf IDs representing the order of the leaf being selected and they will be consistent during tracking.}
\label{fig:alignResult}
\end{centering}
\end{figure*}



\begin{figure*}
\begin{centering}
\begin{tabular}{c c@{} c@{} c@{} c@{} c@{} c@{} c@{} c@{} c@{}}
%\begin{tabular}{lllllllll}
(a) &
\includegraphics[trim= 100 100 150 10, clip, width=.1\textwidth]{Figures/trackExample/1_1}&
\includegraphics[trim= 100 100 150 10, clip, width=.1\textwidth]{Figures/trackExample/1_2}&
\includegraphics[trim= 100 100 150 10, clip, width=.1\textwidth]{Figures/trackExample/1_3}&
\includegraphics[trim= 100 100 150 10, clip, width=.1\textwidth]{Figures/trackExample/1_4}&
\includegraphics[trim= 100 100 150 10, clip, width=.1\textwidth]{Figures/trackExample/1_5}&
\includegraphics[trim= 100 100 150 10, clip, width=.1\textwidth]{Figures/trackExample/1_6}&
\includegraphics[trim= 100 100 150 10, clip, width=.1\textwidth]{Figures/trackExample/1_7}&
\includegraphics[trim= 100 100 150 10, clip, width=.1\textwidth]{Figures/trackExample/1_8}&
\includegraphics[trim= 100 100 150 10, clip, width=.1\textwidth]{Figures/trackExample/1_9}\\
(b)&
\includegraphics[trim= 100 100 150 10, clip, width=.1\textwidth]{Figures/trackExample/2_1}&
\includegraphics[trim= 100 100 150 10, clip, width=.1\textwidth]{Figures/trackExample/2_2}&
\includegraphics[trim= 100 100 150 10, clip, width=.1\textwidth]{Figures/trackExample/2_3}&
\includegraphics[trim= 100 100 150 10, clip, width=.1\textwidth]{Figures/trackExample/2_4}&
\includegraphics[trim= 100 100 150 10, clip, width=.1\textwidth]{Figures/trackExample/2_5}&
\includegraphics[trim= 100 100 150 10, clip, width=.1\textwidth]{Figures/trackExample/2_6}&
\includegraphics[trim= 100 100 150 10, clip, width=.1\textwidth]{Figures/trackExample/2_7}&
\includegraphics[trim= 100 100 150 10, clip, width=.1\textwidth]{Figures/trackExample/2_8}&
\includegraphics[trim= 100 100 150 10, clip, width=.1\textwidth]{Figures/trackExample/2_9}\\
(c) &
\includegraphics[trim= 100 70 120 10, clip, width=.1\textwidth]{Figures/trackExample/3_1.pdf}&
\includegraphics[trim= 100 70 120 10, clip, width=.1\textwidth]{Figures/trackExample/3_2.pdf}&
\includegraphics[trim= 100 70 120 10, clip, width=.1\textwidth]{Figures/trackExample/3_3.pdf}&
\includegraphics[trim= 100 70 120 10, clip, width=.1\textwidth]{Figures/trackExample/3_4.pdf}&
\includegraphics[trim= 100 70 120 10, clip, width=.1\textwidth]{Figures/trackExample/3_5.pdf}&
\includegraphics[trim= 100 70 120 10, clip, width=.1\textwidth]{Figures/trackExample/3_6.pdf}&
\includegraphics[trim= 100 70 120 10, clip, width=.1\textwidth]{Figures/trackExample/3_7.pdf}&
\includegraphics[trim= 100 70 120 10, clip, width=.1\textwidth]{Figures/trackExample/3_8.pdf}&
\includegraphics[trim= 100 70 120 10, clip, width=.1\textwidth]{Figures/trackExample/3_9.pdf}\\

(d) &
\includegraphics[width=.09\textwidth]{Figures/trackExample/4_1}&
\includegraphics[width=.09\textwidth]{Figures/trackExample/4_2}&
\includegraphics[width=.09\textwidth]{Figures/trackExample/4_3}&
\includegraphics[width=.09\textwidth]{Figures/trackExample/4_4}&
\includegraphics[width=.09\textwidth]{Figures/trackExample/4_5}&
\includegraphics[width=.09\textwidth]{Figures/trackExample/4_6}&
\includegraphics[width=.09\textwidth]{Figures/trackExample/4_7}&
\includegraphics[width=.09\textwidth]{Figures/trackExample/4_8}&
\includegraphics[width=.09\textwidth]{Figures/trackExample/4_9}\\

(e) &
\includegraphics[trim= 160 100 160 60, clip, width=.1\textwidth]{Figures/trackExample/5_1}&
\includegraphics[trim= 160 100 160 60, clip, width=.1\textwidth]{Figures/trackExample/5_2}&
\includegraphics[trim= 160 100 160 60, clip, width=.1\textwidth]{Figures/trackExample/5_3}&
\includegraphics[trim= 160 100 160 60, clip, width=.1\textwidth]{Figures/trackExample/5_4}&
\includegraphics[trim= 160 100 160 60, clip, width=.1\textwidth]{Figures/trackExample/5_5}&
\includegraphics[trim= 160 100 160 60, clip, width=.1\textwidth]{Figures/trackExample/5_6}&
\includegraphics[trim= 160 100 160 60, clip, width=.1\textwidth]{Figures/trackExample/5_7}&
\includegraphics[trim= 160 100 160 60, clip, width=.1\textwidth]{Figures/trackExample/5_8}&
\includegraphics[trim= 160 100 160 60, clip, width=.1\textwidth]{Figures/trackExample/5_9}\\

(f) &
\includegraphics[trim= 100 90 120 60, clip, width=.1\textwidth]{Figures/trackExample/6_1}&
\includegraphics[trim= 100 90 120 60, clip, width=.1\textwidth]{Figures/trackExample/6_2}&
\includegraphics[trim= 100 90 120 60, clip, width=.1\textwidth]{Figures/trackExample/6_3}&
\includegraphics[trim= 100 90 120 60, clip, width=.1\textwidth]{Figures/trackExample/6_4}&
\includegraphics[trim= 100 90 120 60, clip, width=.1\textwidth]{Figures/trackExample/6_5}&
\includegraphics[trim= 100 90 120 60, clip, width=.1\textwidth]{Figures/trackExample/6_6}&
\includegraphics[trim= 100 90 120 60, clip, width=.1\textwidth]{Figures/trackExample/6_7}&
\includegraphics[trim= 100 90 120 60, clip, width=.1\textwidth]{Figures/trackExample/6_8}&
\includegraphics[trim= 100 90 120 60, clip, width=.1\textwidth]{Figures/trackExample/6_9}\\


& Day 1 & Day 2 & Day 3 & Day 4 & Day 5 & Day 6 & Day 7 & Day 8 & Day 9 \\
\end{tabular}
\caption{Tracking result for plant $16$ with a first frame for each day. $(a)$ Example frames in fluorescence modality. $(b)$ Manual leaf label results overlaid with tip locations. $(c)$ Leaf tracking results on fluorescence images. $(d)$ Example frames in RGB modality. $(e)$ Label propagation from fluorescence images to RGB images. $(f)$ Leaf tracking results on RGB images.}
\label{fig:trackExample}
\end{centering}
\end{figure*}


\begin{figure*}
\centering
\includegraphics[width=.81\textwidth]{Figures/performance.pdf}\\
\caption{Performance of the baseline method on the testing set of the fluorescence modality of Arabidopsis plant.}
\label{fig:performance}
\end{figure*}


\subsection{Performance and Analysis}
We apply our algorithm to all $135$ frames of each video and evaluate the performance on labeled $36$ frames.
Figure~\ref{fig:alignResult} shows some examples of leaf alignment results on the last frame of each video. 
Our framework works very well on segmenting large leaves with no overlap to neighbor leaves.
For overlapping leaves, it becomes more challenging as the edges in the overlapping area are more difficult to be detected.
However, when the overlapping leaves are further away from the center, they will have a higher chance to be detected as shown in $(c)$ of Figure~\ref{fig:alignResult}.
When the overlapping leaves are close to the center, smaller leaves will be covered by larger leaves as shown in $(a), (d), (e)$ of Figure~\ref{fig:alignResult}.


%Leaf alignment provides the leaf candidates for tracking over time.
Figure~\ref{fig:trackExample} shows the leaf tracking result of one Arabidopsis video in both fluorescence and RGB modalities. 
We can see the high quality leaf label propagation results on this video. 
The leaf template transformation works well for most of the leaves.
As plant grows, younger leaves may grow faster than older leaves and occlude the older leaves.
As shown in Figure~\ref{fig:trackExample} (b)(e), purple leaf replaces the red leaf at day $6$.
%Our backward tracking algorithm tracks leaves from the last frame to the first frame.
The two leaves are still being considered as one leaf (ID $8$) in day $4$ and day $3$ (Figure~\ref{fig:trackExample}(c)).
Leaf $8$ in day $1$ and $2$ is a leaf ID switch w.r.t.~the purple ground truth leaf and will not be considered as a consistently tracked leaf (Figure~\ref{fig:trackExample}(c)).
However, they are still evaluated as well aligned and segmented leaves.
\todo{May add more if the final figure updated}


For quantitative evaluation, we vary $\tau$ from $0$ to $1$ and generate the first three evaluation metrics, as shown in Figure~\ref{fig:performance}.
For both RGB and fluorescence modalities, {\it{ULR}} decreases as $\tau$ increases as more leaves are being considered as matched leaves.
As $\tau$ keeps increasing, {\it{ULR}} approaches a constant value, which is the different number in leaf counting that results from both miss detection and false alarms.
{\it{LE}} increases as $\tau$ increases as it includes leaves with larger tip-based errors for averaging.
{\it{TC}} increases as $\tau$ increases as more leaves are being considered as correctly tracked leaves.
Note that {\it{TC}} is influenced by the length of frames evaluated for each video.
As the longer frames we evaluate, the higher chance tracking will fail.
Overall, our method can detect $87\%$ and track $50\%$ of all labeled leaves with less than $20\%$ average tip-based errors.

We generate a label image for each frame based on the leaf segmentation results and compute the {\it{SBD}} score for each labeled image.
The average {\it{SBD}} is $0.61$ for fluorescence images and $0.56$ for RGB images. 







